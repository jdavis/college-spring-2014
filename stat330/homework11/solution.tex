\documentclass{article}

\usepackage{fancyhdr}
\usepackage{extramarks}
\usepackage{amsmath}
\usepackage{amsthm}
\usepackage{amsfonts}
\usepackage{multirow}
\usepackage{enumerate}

\topmargin=-0.45in
\evensidemargin=0in
\oddsidemargin=0in
\textwidth=6.5in
\textheight=9.0in
\headsep=0.25in

\linespread{1.1}

\pagestyle{fancy}
\lhead{\hmwkAuthorName}
\chead{\hmwkClass\ (\hmwkClassInstructor\ \hmwkClassTime): \hmwkTitle}
\rhead{\firstxmark}
\lfoot{\lastxmark}
\cfoot{\thepage}

\renewcommand\headrulewidth{0.4pt}
\renewcommand\footrulewidth{0.4pt}

\setlength\parindent{0pt}

\newcommand{\enterProblemHeader}[1]{
    \nobreak\extramarks{}{Problem \arabic{#1} continued on next page\ldots}\nobreak{}
    \nobreak\extramarks{Problem \arabic{#1} (continued)}{Problem \arabic{#1} continued on next page\ldots}\nobreak{}
}

\newcommand{\exitProblemHeader}[1]{
    \nobreak\extramarks{Problem \arabic{#1} (continued)}{Problem \arabic{#1} continued on next page\ldots}\nobreak{}
    \stepcounter{#1}
    \nobreak\extramarks{Problem \arabic{#1}}{}\nobreak{}
}

\setcounter{secnumdepth}{0}
\newcounter{partCounter}
\newcounter{homeworkProblemCounter}

% Start at problem 8 from homework 10
\setcounter{homeworkProblemCounter}{8}
\nobreak\extramarks{Problem \arabic{homeworkProblemCounter}}{}\nobreak{}

\newenvironment{homeworkProblem}{
    \section{Problem \arabic{homeworkProblemCounter}}
    \setcounter{partCounter}{1}
    \enterProblemHeader{homeworkProblemCounter}
}{
    \exitProblemHeader{homeworkProblemCounter}
}

\newcommand{\hmwkTitle}{Homework\ \#11}
\newcommand{\hmwkDueDate}{April 30, 2014}
\newcommand{\hmwkClass}{Stat330}
\newcommand{\hmwkClassTime}{Section A}
\newcommand{\hmwkClassInstructor}{Mr. Lanker}
\newcommand{\hmwkAuthorName}{Josh Davis}

\title{
    \vspace{2in}
    \textmd{\textbf{\hmwkClass:\ \hmwkTitle}}\\
    \normalsize\vspace{0.1in}\small{Due\ on\ \hmwkDueDate\ at 3:10pm}\\
    \vspace{0.1in}\large{\textit{\hmwkClassInstructor\ \hmwkClassTime}}
    \vspace{3in}
}

\author{\textbf{\hmwkAuthorName}}
\date{}

\newcommand{\deriv}[1]{\frac{\mathrm{d}}{\mathrm{d}x} (#1)}
\newcommand{\pderiv}[2]{\frac{\partial}{\partial #1} (#2)}
\newcommand{\dx}{\mathrm{d}x}
\newcommand{\solution}{\textbf{\large Solution}}

\newcommand{\E}{\mathrm{E}}
\newcommand{\Var}{\mathrm{Var}}
\newcommand{\Cov}{\mathrm{Cov}}
\newcommand{\Bias}{\mathrm{Bias}}
\newcommand{\Std}{\mathrm{Std}}
\newcommand{\dist}[1]{\sim \mathrm{#1}}
\newcommand{\pval}{\(p\)-value}
\newcommand{\tstat}{\(t\)-statistic}
\newcommand{\zstat}{\(z\)-statistic}
\newcommand{\tdist}{\(t\)-distribution}
\newcommand{\Likelihood}{\mathcal{L}}

\renewcommand{\part}[1]{\textbf{\large Part \Alph{partCounter}}\stepcounter{partCounter}\\}

\begin{document}

\maketitle

\pagebreak

\begin{homeworkProblem}
    An experimenter tested for differences in attitudes toward smoking before
    and after a film on lung cancer was shown. The experimenter tested to see
    if there was a difference between attitudes that people held about smoking
    before and after viewing the film (either less or more favorable).  She
    found a difference which was significant between the 0.02 and 0.05 levels.
    \\

    \part

    Let \(\mu_1\) and \(\mu_2\) represent the mean attitude towards smoking before
    viewing the film and after viewing the film, respectively. What are the assumed
    hypotheses (null and alternative)?
    \\

    \solution

    The hypotheses are as follows:
    \[
        H_0 : \mu_1 = \mu_2
        \quad
        vs.
        \quad
        H_a : \mu_1 \neq \mu_2
    \]

    \part

    What level of significance indicates the greater degree of significance,
    0.05 or 0.02, i.e. for which level of significance will the experimenter be
    more confidence in rejecting null hypothesis in favor of the alternative?
    \\

    \solution

    A lower value is better as it indicates a smaller possibility of the value
    being that extreme.
    \\

    \part

    If her \(\alpha\) level is 0.05, will she reject \(H_0\) in favor of
    \(H_a\)?
    \\

    \solution

    Yes, if her significance is between 0.05 and 0.02 and her \(\alpha\) level is
    0.05, then that means that the level of significance is less than
    \(\alpha\) which is when we reject the \(H_0\).
    \\

    \part

    Will she reject \(H_0\) in favor of \(H_a\) if she employs the 0.01 level?
    \\

    \solution

    No, since our value is between 0.05 and 0.02, an \(\alpha\) value of 0.01
    is too small and we can't reject \(H_0\).
\end{homeworkProblem}

\pagebreak

\begin{homeworkProblem}
    The mean yield of corn in the US is about 120 bushels per acre (from 1989).
    A survey of 50 farmers this year gives a sample mean yield of \(\bar{x} =
    123.6\) bushels per acre.  We want to know whether this is good evidence
    that the national mean this year is not 120 bushels per acre.  Assume that
    the sample is i.i.d. from the entire population and that the standard
    deviation of the yield in this population is \(\sigma = 10\) bushels per
    acre.
    \\

    Give the {\pval} for the test of
    \[
        H_0 : \mu = 120
        \quad
        vs.
        \quad
        H_a : \mu \neq 120
    \]

    Are you convinced that the population mean is not 120 bushels per acre? Use
    the 0.05 significance level in making your decision.
    \\

    \solution

    First let's calculate the {\zstat}, which gives us:
    \[
        Z
        = \frac{
            \bar{X} - \mu
        }{
            \sigma / \sqrt{n}
        }
        = \frac{
            123.6 - 120
        }{
            10 / \sqrt{50}
        }
        = \frac{
            123.6 - 120
        }{
            10 / \sqrt{50}
        }
        = 2.55
    \]

    This gives us a {\pval} of 0.015. Since this value is less than \(0.05\),
    we have moderate evidence that we can reject the null hypothesis and thus
    we accept that we have evidence that \(\mu \neq 120\).
\end{homeworkProblem}

\pagebreak

\begin{homeworkProblem}
    In the past, the mean score of the seniors at South High on the ACT exam
    has been 20.0.  This year a special preparation course is offered, and all
    43 seniors planning to take the ACT enroll in the course. The mean of their
    ACT scores is 21.1. Assume that the ACT scores vary normally with \(\sigma
    = 6\).
    \\

    Is the outcome good evidence that this class's true mean is not 20? State
    your hypotheses, compute the {\pval}, and assess the amount of evidence.
    \\

    \solution

    Solution.
\end{homeworkProblem}

\pagebreak

\begin{homeworkProblem}
    A computer has a random number generator designed to produce random numbers that are uniformly
    distributed within the interval from 0 to 1. If this is true, the numbers come from a population
    with \(\mu = \frac{1}{2}\) and \(\sigma^2 = \frac{1}{12}\).
    \\

    A command to generate one million random numbers results in a sample mean
    of 0.4992894. Assume that the population variance remains fixed. We want to
    use the results of this experiment to test if \(\mu\) is in fact one-half.
    \\

    \part

    State the hypotheses for this test.
    \\

    \solution

    Solution.
    \\

    \part

    Calculate the value of the {\zstat}.
    \\

    \solution

    Solution.
    \\

    \part

    Compute the {\pval}.
    \\

    \solution

    Solution.
    \\

    \part

    Is the result significant at the \(\alpha = 0.05\) level?
    \\

    \solution

    Solution.
    \\

    \part

    Is the result significant at the \(\alpha = 0.01\) level?
    \\

    \solution

    Solution.
    \\

    \part

    I performed this test in R, which has a decent random number generator. Are
    you surprised about the result of this experiment? Why or why not?
    \\

    \solution

    Solution.
    \\

    \part

    \textbf{Extra Credit:} Now what if I told you that I generated 100 sets of
    one million random numbers and used the lowest sample mean of the 100 sets
    for this problem, now are you surprised about the result? Why or why not?
    \\

    \solution

    Solution.
\end{homeworkProblem}


\end{document}
